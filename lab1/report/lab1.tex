\include{settings}

\begin{document}

\include{title}

\tableofcontents
\newpage

\section{Программа работы}

\begin{enumerate}
	\item Реализовать линейное растяжение гистограммы.
	\item Реализовать эквализацию гистограммы.
	\item Реализовать приведение гистограммы к заданной.
\end{enumerate}

\section{Выполнение работы}

\subsection{Линейное растяжение гистограммы}

Значение пикселя $p_{in}$ преобразуется в $p_{out}$ по формуле:
$$
p_{out} = (p_{in} - I_{low}) \cdot \frac{O_{high} - O_{low}}{I_{high} - I_{low}} + O_{low},
$$
где $I$ -- исходное изображение, а $O$ -- результирующее изображение.

При растяжении гистограммы используем $O_{high} = 255$ и $O_{low} = 0$, тогда
$$
p_{out} = (p_{in} - I_{low}) \cdot \frac{255}{I_{high} - I_{low}}
$$

Реализуем растяжение гистограммы в более общем виде, где $O_{high}$ и $O_{low}$ могут быть заданы через параметры.

\begin{lstlisting}
def stretch(img, low=0, high=K):
	img_low = img.min()
	img_high = img.max()
	return ((img - img_low) / (img_high - img_low) * (high - low) + low).astype('uint8')
\end{lstlisting}

Применим растяжение гистограммы к тестовым изображениям. Дополнительно выведем гистограмму исходного и получившегося изображения.

\begin{figure}[H]
	\centering
	\includegraphics[width=\linewidth]{tiger_stretch}
	\caption{Линейное растяжение гистограммы (1)}
\end{figure}

\begin{figure}[H]
	\centering
	\includegraphics[width=\linewidth]{lion_stretch}
	\caption{Линейное растяжение гистограммы (2)}
\end{figure}

\begin{figure}[H]
	\centering
	\includegraphics[width=\linewidth]{wolf_stretch}
	\caption{Линейное растяжение гистограммы (3)}
\end{figure}

\subsection{Устойчивое линейное растяжение гистограммы}

Попробуем применить устойчивое линейное растяжение гистограммы: будем отбрасывать $M\%$ (например, $5\%$) самых темных и самых светлых пикселей при подсчете минимума и максимума гистограммы исходного изображения. Это позволяет более устойчиво применить растягивание гистограммы, когда слишком светлых или слишком светлых пикселей небольшое количество.

Реализуем функцию, для определения минимального и максимального значения пикселя в исходном изображении, которое должно быть сохранено. Затем создадим функцию для применения устойчивого линейного растяжения гистограммы, которая будет принимать эти значения на вход.

\begin{lstlisting}
def find_min_and_max(img, hist, drop=0.05):    
	k = int(drop * img.size)
	x_min, x_max = 0, K
	
	cnt = 0
	while x_min < x_max:
		if cnt >= k:
			break
		if hist[x_min] < hist[x_max]:
			cnt += hist[x_min]
			x_min += 1
		else:
			cnt += hist[x_max]
			x_max -= 1

	return x_min, x_max

def stable_stretch(img, img_low, img_high, low=0, high=255):
	img = img.astype('float')
	corrected = (img - img_low) * (high - low) / (img_high - img_low)
	res = np.clip(corrected, low, high)
	return res.astype('uint8')
\end{lstlisting}

Применим алгоритм к изображениям. Дополнительно на гистограмме исходного изображения выделим вычисленные границы значений пикселей исходного изображения.

\begin{figure}[H]
	\centering
	\includegraphics[width=\linewidth]{tiger_stretch_stable}
	\caption{Устойчивое линейное растяжение гистограммы (1)}
\end{figure}

\begin{figure}[H]
	\centering
	\includegraphics[width=\linewidth]{lion_stretch_stable}
	\caption{Устойчивое линейное растяжение гистограммы (2)}
\end{figure}

\begin{figure}[H]
	\centering
	\includegraphics[width=\linewidth]{wolf_stretch_stable}
	\caption{Устойчивое линейное растяжения гистограммы (3)}
\end{figure}

\subsection{Эквализация гистограммы}

Применим другой метод повышения контрастности изображения -- эквализацию гистограммы. Определим функцию распределения $\mathit{cdf}(n) = h(0) + h(1) + ... + \mathit{cdf}(n)$. Другими словами, функция распределения является кумулятивной гистограммой. Наша задача сводится к тому, чтобы функция распределения имела вид, близкий к линейному, тогда пиксели изображения будут более равномерно использовать весь диапазон значений. Формула для преобразования пикселя входного изображения $p_{in}$:
$$
p_{out} = \mathit{round} \left( \frac{\mathit{cdf}(p_{in}) - \mathit{cdf}_{min}}{N} \cdot 255 \right),
$$
где $N$ -- общее число пикселей в изображении.

Реализуем функцию для нахождения кумулятивной суммы по рассчитанной гистограмме изображения. Затем применим ее для эквализации гистограммы.

\begin{lstlisting}
def my_cumsum(hist):
	cdf = hist.copy()
	for i in np.arange(1, hist.size):
		cdf[i] = cdf[i - 1] + cdf[i]
	return cdf

np.all(my_cumsum(my_hist(tiger)) == np.cumsum(my_hist(tiger)))
## True

def equalize(x, cdf, cdf_min):
	N = cdf[-1]
	return np.round((cdf[x] - cdf_min) / N * K).astype('uint8')
\end{lstlisting}

Применим созданную функцию для эквализации гистограмм тестовых изображений.

\begin{figure}[H]
	\centering
	\includegraphics[width=\linewidth]{tiger_equalized}
	\caption{Эквализация гистограммы (1)}
\end{figure}

\begin{figure}[H]
	\centering
	\includegraphics[width=\linewidth]{lion_equalized}
	\caption{Эквализация гистограммы (2)}
\end{figure}

\begin{figure}[H]
	\centering
	\includegraphics[width=\linewidth]{wolf_equalized}
	\caption{Эквализация гистограммы (3)}
\end{figure}

\subsection{Приведение гистограммы}

В том случае, если у нас есть референсное изображение, мы можем использовать его гистограмму для преобразования входного изображения. Для этого создадим отображение каждого значения входного изображения в выходное (всего 256 возможных входных и выходных значений), после чего отобразим значение каждого пиксель входного изображения в выходное. 

Рассчитаем гистограммы и функции распределений входного ($\mathit{cdf_1}$) и референсного ($\mathit{cdf_2}$) изображений, после чего найдем такие значения пикселей $p_1$ и $p_2$, что:
$$
\mathit{cdf_1}(p_1) = \mathit{cdf_2}(p_2),
$$
тогда значение пикселя $p_1$ отображается в значение $p_2$.

Реализуем функцию приведения гистограммы.

\begin{lstlisting}
def match_histogram(img, ref):
	res = img.copy()
	
	img_hist = my_hist(img)
	img_cdf = my_cumsum(img_hist)
	img_cdf_norm = img_cdf / img.size
	
	ref_hist = my_hist(ref)
	ref_cdf = my_cumsum(ref_hist)
	ref_cdf_norm = ref_cdf / ref.size
	
	mapping = np.zeros(K + 1)
	for i in np.arange(K + 1):
		j = K
		while True:
			mapping[i] = j
			j = j - 1
			if j < 0 or img_cdf_norm[i] > ref_cdf_norm[j]:
				break
	
	for i in np.arange(res.shape[0]):
		for j in np.arange(res.shape[1]):
			a = res.item(i,j)
			b = mapping[a]
			res.itemset((i,j), b)
	
	return res
\end{lstlisting}


\begin{figure}[H]
	\centering
	\includegraphics[width=\linewidth]{tiger_lion_matched}
	\caption{Приведение гистограммы (1)}
\end{figure}

\begin{figure}[H]
	\centering
	\includegraphics[width=\linewidth]{lion_wolf_matched}
	\caption{Приведение гистограммы (2)}
\end{figure}

\section{Выводы}

В данной работе были реализованы различные операции над гистограммой изображения:

\begin{itemize}
	\item линейное растяжение и устойчивое линейное растяжение гистограммы;
	\item эквализация гистограммы;
	\item приведение гистограммы изображения к заданному виду.
\end{itemize}

\end{document}