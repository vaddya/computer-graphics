\include{settings}

\begin{document}

\include{title}

\tableofcontents
\newpage

\section{Программа работы}

\begin{enumerate}
	\item Реализовать линейное растяжение гистограммы.
	\item Реализовать эквализацию гистограммы.
	\item Реализовать приведение гистограммы к заданной.
\end{enumerate}

\section{Выполнение работы}

\subsection{Линейное растяжение гистограммы}

Значение пикселя $p_{in}$ преобразуется в $p_out$ по формуле:
$$
p_{out} = (p_{in} - I_{low}) \cdot \frac{O_{high} - O_{low}}{I_{high} - I_{low}} + O_{low},
$$
где $I$ -- исходное изображение, а $O$ -- результирующее изображение.

При растяжении гистограммы используем $O_{high} = 255$ и $O_{low} = 0$, тогда
$$
p_{out} = (p_{in} - I_{low}) \cdot \frac{255}{I_{high} - I_{low}}
$$

\begin{figure}[H]
	\centering
	\includegraphics[width=\linewidth]{tiger_stretch}
	\caption{Линейное растяжение гистограммы (1)}
\end{figure}

\begin{figure}[H]
	\centering
	\includegraphics[width=\linewidth]{lion_stretch}
	\caption{Линейное растяжение гистограммы (2)}
\end{figure}

\begin{figure}[H]
	\centering
	\includegraphics[width=\linewidth]{wolf_stretch}
	\caption{Линейное растяжение гистограммы (3)}
\end{figure}

Попробуем применить устойчивое линейное растяжение гистограммы: будем отбрасывать $M\%$ (например, $5\%$) самых темных и самых светлых пикселей при подсчете минимума и максимума гистограммы исходного изображения. Это позволяет более устойчиво применить растягивание гистограммы, когда слишком светлых или слишком светлых пикселей небольшое количество.

\begin{figure}[H]
	\centering
	\includegraphics[width=\linewidth]{tiger_stretch_stable}
	\caption{Устойчивое линейное растяжение гистограммы (1)}
\end{figure}

\begin{figure}[H]
	\centering
	\includegraphics[width=\linewidth]{lion_stretch_stable}
	\caption{Устойчивое линейное растяжение гистограммы (2)}
\end{figure}

\begin{figure}[H]
	\centering
	\includegraphics[width=\linewidth]{wolf_stretch_stable}
	\caption{Устойчивое линейное растяжения гистограммы (3)}
\end{figure}

\subsection{Эквализация гистограммы}

Применим другой метод повышения контрастности изображения -- эквализацию гистограммы. Определим функцию распределения $\mathit{cdf}(n) = h(0) + h(1) + ... + \mathit{cdf}(n)$. Другими словами, функция распределения является кумулятивной гистограммой. Наша задача сводится к тому, чтобы функция распределения имела вид, близкий к линейному, тогда пиксели изображения будут более равномерно использовать весь диапазон значений. Формула для преобразования пикселя входного изображения $p_{in}$:

$$
p_{out} = \mathit{round} \left( \frac{\mathit{cdf}(p_{in}) - \mathit{cdf}_{min}}{N} \cdot 255 \right),
$$

где $N$ -- общее число пикселей в изображении.

\begin{figure}[H]
	\centering
	\includegraphics[width=\linewidth]{tiger_equalized}
	\caption{Эквализация гистограммы (1)}
\end{figure}

\begin{figure}[H]
	\centering
	\includegraphics[width=\linewidth]{lion_equalized}
	\caption{Эквализация гистограммы (2)}
\end{figure}

\begin{figure}[H]
	\centering
	\includegraphics[width=\linewidth]{wolf_equalized}
	\caption{Эквализация гистограммы (3)}
\end{figure}

\subsection{Приведение гистограммы}

В том случае, если у нас есть референсное изображение, мы можем использовать его гистограмму для преобразования входного изображения. Для этого создадим отображение каждого значения входного изображения в выходное (всего 256 возможных входных и выходных значений), после чего отобразим значение каждого пиксель входного изображения в выходное. 

Рассчитаем гистограммы и функции распределений входного ($\mathit{cdf_1}$) и референсного ($\mathit{cdf_2}$) изображений, после чего найдем такие значения пикселей $p_1$ и $p_2$, что:
$$
\mathit{cdf_1}(p_1) = \mathit{cdf_2}(p_2),
$$
тогда значение пикселя $p_1$ отображается в значение $p_2$.

\begin{figure}[H]
	\centering
	\includegraphics[width=\linewidth]{tiger_lion_matched}
	\caption{Приведение гистограммы (1)}
\end{figure}

\begin{figure}[H]
	\centering
	\includegraphics[width=\linewidth]{lion_wolf_matched}
	\caption{Приведение гистограммы (2)}
\end{figure}

\section{Выводы}

В данной работе были реализованы различные операции над гистограммой изображения:
\begin{itemize}
	\item Линейное растяжение и устойчивое линейное растяжение гистограммы;
	\item Эквализация гистограммы;
	\item Приведение гистограммы изображения к заданному виду.
\end{itemize}

\end{document}